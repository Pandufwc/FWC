\documentclass[12pt]{article}
\usepackage{graphicx}
%\documentclass[journal,12pt,twocolumn]{IEEEtran}
\usepackage[none]{hyphenat}
\usepackage{graphicx}
\usepackage{blindtext}
\usepackage{multicol}
\usepackage{listings}
\usepackage[english]{babel}
\usepackage{graphicx}
\usepackage{caption}
\usepackage[parfill]{parskip}
\usepackage{hyperref}
\usepackage{gensymb}
\usepackage{commath}
\usepackage{amssymb}
\usepackage{amsthm}
\usepackage{tikz}
\usepackage{booktabs}
\usepackage{tabularx}
\usepackage{multirow}
%\usepackage{setspace}\doublespacing\pagestyle{plain}
\def\inputGnumericTable{}
\usepackage{color}                                            %%
    \usepackage{array}                                            %%
    \usepackage{longtable}                                        %%
    \usepackage{calc}                                             %%
    \usepackage{multirow}                                         %%
    \usepackage{hhline}                                           %%
    \usepackage{ifthen}
\usepackage{array}
\usepackage{amsmath}   % for having text in math mode
\usepackage{parallel,enumitem}
\usepackage{listings}
\lstset{
language=tex,
frame=single,
breaklines=true
}
 
%Following 2 lines were added to remove the blank page at the beginning
\usepackage{atbegshi}% http://ctan.org/pkg/atbegshi
\AtBeginDocument{\AtBeginShipoutNext{\AtBeginShipoutDiscard}}
%
%New macro definitions
\newcounter{matchleft}\newcounter{matchright}

\newenvironment{matchtabular}{%
  \setcounter{matchleft}{0}%
  \setcounter{matchright}{0}%
  \tabularx{\textwidth}{%
    >{\leavevmode\hbox to 1.5em{\stepcounter{matchleft}\arabic{matchleft}.}}X%
    >{\leavevmode\hbox to 1.5em{\stepcounter{matchright}\alph{matchright})}}X%
    }%
}{\endtabularx}
\newcommand{\mydet}[1]{\ensuremath{\begin{vmatrix}#1\end{vmatrix}}}
\providecommand{\brak}[1]{\ensuremath{\left(#1\right)}}
\providecommand{\norm}[1]{\left\lVert#1\right\rVert}
\newcommand{\solution}{\noindent \textbf{Solution: }}
\newcommand{\myvec}[1]{\ensuremath{\begin{pmatrix}#1\end{pmatrix}}}
\providecommand{\abs}[1]{\left\vert#1\right\vert}
\let\vec\mathbf
\begin{document}
\begin{center}
\enlargethispage{-4cm}
\title{\textbf{Conic Sections}}
\date{\vspace{-5ex}} %Not to print date automatically
\maketitle
\end{center}
\setcounter{page}{1}
\section*{11$^{th}$ Maths - Chapter 11}
\begin{center}
\textbf{EXERCISE 10.1}
\end{center}
Choose the corret answer frow the given four options :
\begin{enumerate}
\item To divide a line segment $AB$ is the ratio 5:7, first a ray $AB$ is drawn so that $\angle{BAX}$ is an acute angle and then at equal distances points are marked on the ray $AX$ such that the minimum number of these points is 
\begin{enumerate}
\item 8
\item 10
\item 11
\item 12
\end{enumerate}
\item To divide a line segment $AB$ in the ratio 4:7, a ray $AX$ is drawn first such that $\angle{BAX}$ is an acute angle and then points $\vec{A_1}$, $\vec{A_2}$, $\vec{A_3}$,.... are locatad at equal distances on the ray $AX$ and the point $\vec{B}$ is joined to 
\begin{enumerate}
	\item $\vec{A}_{12}$
	\item $\vec{A}_{11}$
	\item $\vec{A}_{10}$
	\item $\vec{A}_9$
\end{enumerate}
\item To divide a line segment $AB$ in ratio 5:6, draw a ray $AX$ such that $\angle{ABX}$ is an acute angle, then draw a ray $BY$ parallel to $AX$ and the points $A_1$, $A_2$, $A_3$, ... and $B_1$, $B_2$, $B_3$, ... are located at equal distances on ray $AX\text{ and }BY$, respectively, Then the points joinied are 
\begin{enumerate}
\item $\vec{A}_5\text{ and }\vec{B}_6$
\item $\vec{A}_6\text{ and }\vec{B}_5$
\item $\vec{A}_4\text{ and }\vec{B}_5$
\item $\vec{A}_5\text{ and }\vec{B}_4$
\end{enumerate}
\item To construct a triangle similar to a given $\triangle{ABC}$ with its sides $\frac{3}{7}$ of the corresponding sides of $\triangle{ABC}$, fist draw a ray $BX$ such that $\angle{CBX}$ is an acute angle and $x$ lies on 
the opposite side of $\vec{A}$ with respect to $BC$. Then locate points $B_1$, $B_2$, $B_3$, ... on $BX$ at equal distances and next step is to join
\begin{enumerate}
	\item $\vec{B}_{10}\text{ to }\vec{C}$
\item $\vec{B}_3\text{ to }\vec{C}$ 
\item $\vec{B}_7\text{ to }\vec{C}$
\item $\vec{B}_4\text{ to }\vec{C}$
\end{enumerate}
\item To construct a triangle similar to a given $\triangle{ABC}$ with its sides $\frac{8}{5}$ of the corresponding sides of $\triangle{ABC}$ draw a ray $BX$ such that $\angle{CBX}$ is an acute angle and $X$ is on the opposite side of $\vec{A}$ with respect to $BC$. The minimum number of points to be located at equal distances on ray $BX$ is 
\begin{enumerate}
\item 5
\item 8
\item 13
\item 3
\end{enumerate}
\item To draw a pair of tangents to a circle which are inclined to each other at an angle of $60\degree$, it is required
to draw tangents at end points of those two radii of the circle, the angle between them should be 
\begin{enumerate}
\item $135\degree$
\item $90\degree$
\item $60\degree$
\item $120\degree$ 
\end{enumerate}
\end{enumerate}
\begin{center}                            \textbf{EXERCISE 10.2}                    \end{center}
Write True or False and give reasons for your answer in each of the following :
\begin{enumerate}
	\item By geometrical construction, it is possible to divide a line segment in the $\sqrt{3}$: $\frac{1}{\sqrt{3}}$.
\item To construct a triangle similar to a given $\triangle{ABC}$ with its sides $\frac{7}{3}$ of the corresponding sides of $\triangle{ABC}$, draw a ray $BX$ making acute angle with $BC$ and $x$ lies on 
the opposite side of $\vec{A}$ with respect to $BC$. The points $B_1$, $B_2$, ...., $B_7$ are located at equal distances on $BX$, $B_3$ is joined to $\vec{c}$ and then a line segment $B_6C$' is drawn produced. Final line segment $A'C'$ is drawn parallel to $AC$.
\item A pair of tangents can be constructed from a point $\vec{p}$ to a circle of radius 3.5 cm situated at a distance of 3 cm from the centre.
\item A pair of tangents can be constructed to a circle inclined at an angle of $170\degree$.
\end{enumerate}
\begin{center}                            \textbf{EXERCISE 10.3}                    \end{center}
	\begin{enumerate}
\item Draw a line segment of length 7 cm. Find a point $\vec{P}$ on it which divides it in the ratio 3:5.
\item Draw a right triangle ${ABC}$ n which $BC=12$ cm, $AB=5$ cm and $\angle{B}=90\degree$.
Construct a triange similar to it and of scale factor $\frac{2}{3}$. Is the new triangle also a right triangle ?
\item Draw a triangle ${ABC}$ in which $BC=6$ cm, $CA=5$ cm and $AB=4$ cm. Construct a triangle similar to it and of scale factor $\frac{5}{3}$.
\item Construct a tangent to a circle of radius 4 cm from a point which is at a distance of 6 cm from its centre.
	\end{enumerate}
	\begin{center}                            \textbf{EXERCISE 10.4}                    \end{center}
		\begin{enumerate}
\item Two line segments ${AB}\text{ and }{AC}$ include an angle of $60\degree$ where $AB=5$ cm and $AC=7$ cm, respectively such that $AP=\frac{3}{4}AB\text{ and }AQ=\frac{1}{4}AC$. Join $\vec{P}\text{ and }\vec{Q}$ and measure the length $PQ$.
\item Draw a parallelogram ${ABCD}$ in which $BC=5$ cm, $AB=3$ cm and $\angle{ABC}=60\degree$, divide it into triangles ${ACB}\text{ and }{ABD}$ by the diagonal $BD$. 
Construct the triangle $BD'C'$ similar to $\triangle{BDC}$ with scale factor $\frac{4}{3}$. Draw the line segment $D'A'$ parallel to $DA$ where $\vec{A}$' lies on extended side $BA$. Is $A'BC'D'$ a parallelogram? 
\item Draw two concentric circles of radii 3 cm and 5 cm. Taking a point on outer circle construct the pair of tangents to the other. Measure the length of a tangent and verify it by actual calculation.
\item Draw an isosceles triangle ${ABC}$ in which $AB$=$AC$=6 cm and $BC$ =6 cm. Construct a triangle $PQR$ similar to $\triangle{ABC}$in which $PQ$=8 cm. Also justify the construction.
\item Draw a triangle ${ABC}$ in which $AB$=5 cm. $BC=6 cm\text{ and }\angle {ABC}=60\degree$. Construct a triangle similar to $\triangle{ABC}$ with scale factor $\frac{5}{7}$. Justify the construction.
\item Draw a circle of radius 4 cm .Construct a pair of tangents to it, the angle detween which is $60\degree$. Also justify the construction. Measure the distance between the centre of the circle and the point of intersection of tangents.
\item Draw a triangle ${ABC}$ in which $AB$=4 cm, $BC=6 cm\text{ and }AC=9$. Construct a triangle similar to $\triangle{ABC}$ with scale factor $\frac{3}{2}$. justify the constrution. Are the two triangles congruent? Note that all the three angles and two sides of the two triangles are equal.

\end{enumerate}
\end{document}
