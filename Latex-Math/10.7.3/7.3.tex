
\documentclass[12pt]{article}
\usepackage{graphicx}
%\documentclass[journal,12pt,twocolumn]{IEEEtran}
\usepackage[none]{hyphenat}
\usepackage{graphicx}
\usepackage{listings}
\usepackage[english]{babel}
\usepackage{graphicx}
\usepackage{caption}
\usepackage{hyperref}
\usepackage{booktabs}
\usepackage{array}
\usepackage{amsmath}   % for having text in math mode
\usepackage{listings}
\lstset{
  frame=single,
  breaklines=true
}
%New macro definitions
\newcommand{\mydet}[1]{\ensuremath{\begin{vmatrix}#1\end{vmatrix}}}
\providecommand{\brak}[1]{\ensuremath{\left(#1\right)}}
\providecommand{\norm}[1]{\left\lVert#1\right\rVert}
\newcommand{\solution}{\noindent \textbf{Solution: }}
\newcommand{\myvec}[1]{\ensuremath{\begin{pmatrix}#1\end{pmatrix}}}
\let\vec\mathbf

\begin{document}
\begin{center}
\textbf\large{CHAPTER-7 \\ COORDINATE GEOMETRY}
\end{center}

\section*{EXERCISE - 7.3}
State whether the following statements are true false. Justify your answer
\begin{enumerate}\item Name the type of triangle formed by the points $\vec{A}(-5,6),\vec{B}(-4,-2),\text{ and }\vec{C}(7,5)$.
\item Find the points on the $x$-axis which are at a distance on $2\sqrt{5}$ from the point$ (7,-4).$ How many such points are there?
\item What type of a quadrilateral do the points $\vec{A}(2,-2),\vec{B}(7,3),\vec{C}(11,-1),\text{ and }\vec{D}(6,-6)$ taken in that order,form?
\item Find the value of a, if the if the distance between the points $\vec{A}(-3,-14) \text{ and }{B}(a,-5)$ is 9 units.
\item Find a point which is equidistant from the points $\vec{A}(-5,4) \text{ and }(-1,6)$ ? How many such points are there ?
\item Find the coordinates of the point $\vec{Q}$ on the $x$-axi which lies on the perpendicular bisector of the line segment joining the points $\vec{A}(-5,-2) \text{ and }{B}(4,-2)$.Name the type of triangle formed by points $\vec{Q},\vec{A}\text{ and }\vec{B}$.
\item Find the value of $m$ if the points $(5,1),(-2,-3) \text{ and }(8,2m)$ are collinear.
\item If the point $\vec{A}(2,-4)$ is equidistant from $\vec{P}(3,8) \text{ and }\vec{Q}(-10,y)$, find the values of $y$, Also find distance $\vec{PQ}$.
\item Find the area of the triangle whose vertices are $(-8,4),(-6,6)\text{ and }(-3,9)$.
\item In what ratio does the $x$-axis divide the line segment joining the points $(-4,-6)\text{ and }(-1,7)$? Find the coordinats of the point of division.
\item Find the ratio in which the point $\vec{P}\brak{\frac{3}{4},\frac{5}{12}}$ divides the line segment joining the points $\vec{A}\brak{\frac{1}{2},\frac{3}{2}}\text{ and }{B}(2,-5)$.
\item If $\vec{P}(9a-2,-b)$ divides line segment joining $\vec{A}(3a+1,-3)\text{ and }\vec{B}(8a,5)$ in the ratio 3:1,find the values of $a$ and $b$.
\item If $(a,b)$ is the mid-point of the line segment joining the point $\vec{A}(10,-6)\text{ and }\vec{B}(k,4)$ and $a-2b=18$, find the value of and the distance $\vec{AB}$.
\item The centre of a circle is $(2a,a-7)$. Find the values of $a$ if the circle passes through the point $(11,-9)$ and has diameter $10\sqrt{2}$ units.
\item The line segment joining the points $\vec{A}(3,2)\text{ and }\vec{B}(5,1)$ is divided at the point $\vec{P}$ in the ratio 1:2 and it lies $3x-18y+k=0$, Find the value of k  
\item If $\vec{D}\brak{\frac{-1}{2},\frac{5}{2}},\vec{E}(7,3)\text{ and }\vec{F}\brak{\frac{7}{2},\frac{7}{2}}$ are the midpoints of sides of $\triangle \vec{ABC}$, find the area of the $\triangle \vec{ABC}$.
\item The points $\vec{A}(2,9),\vec{B}(a,5) \text{ and }\vec{C}(5,5)$ are the verices of a triangle $\vec{ABC}$ right angled at $\vec{B}$. Find the values of a and hence the area of $\triangle \vec{ABC}$.
\item Find the coordinates of the point $\vec{R}$ on the line segment joining the points $\vec{P}(-1,3)\text{ and }\vec{Q}(2,5)$ such that $\vec{PR}={\frac{3}{5}}\vec{PQ}$.
\item Find the velues of k if the points $\vec{A}(k+1,2k),\vec{B}(3k,2k+3)\text{ and }\vec{C}(5k-1,5k)$ are collinear
\item Find the ratio in which the line 2x+3y-5=0 divides the line segment joining the points $(8,-9)\text{ and }(2,1)$. Also find the coordinates of the point of division,
 
\end{enumerate}
\end{document}
