\documentclass[12pt]{article}
\usepackage{graphicx}
%\documentclass[journal,12pt,twocolumn]{IEEEtran}
\usepackage[none]{hyphenat}
\usepackage{graphicx}
\usepackage{blindtext}
\usepackage{multicol}
\usepackage{listings}
\usepackage[english]{babel}
\usepackage{graphicx}
\usepackage{caption}
\usepackage[parfill]{parskip}
\usepackage{hyperref}
\usepackage{gensymb}
\usepackage{commath}
\usepackage{amssymb}
\usepackage{amsthm}
\usepackage{tikz}
\usepackage{booktabs}
\usepackage{tabularx}
\usepackage{multirow}
%\usepackage{setspace}\doublespacing\pagestyle{plain}
\def\inputGnumericTable{}
\usepackage{color}                                            %%
    \usepackage{array}                                            %%
    \usepackage{longtable}                                        %%
    \usepackage{calc}                                             %%
    \usepackage{multirow}                                         %%
    \usepackage{hhline}                                           %%
    \usepackage{ifthen}
\usepackage{array}
\usepackage{amsmath}   % for having text in math mode
\usepackage{parallel,enumitem}
\usepackage{listings}

\begin{document}
\begin{center}
\textbf{CHAPTER-8}
\end{center}

\begin{enumerate}
\item Find the area of the region bounded by the curves $y^2 = 9x$, $y = 3x$.
\item Find the area of the region bounded by the parabola $y^2 = 2px$, $x^2 = 2py$.
\item Find the area of the region bounded by the curve $y = x^3\text{ and }y = x + 6\text{ and }x = 0$.
\item Find the area of the region bounded by the curve $y^2 = 4x$, $x^2 = 4y$.
\item Find the area of the region included between $y^2 = 9x\text{ and }y =x$
\item Find the area of the region enclosed by the parabola $x^2 = y$ and the line $y = x + 2$
\item Find the area of region bounded by the line $x = 2$ and the parabola $y^2 = 8x$
\item Sketch the region ${(x,0) : y = \sqrt{4 - x^2}}$ and $x$-axis. Find the area of the region using integration.
\item Calcualte the area under the curve $y = 2\sqrt{x}$ included between the lines $x = 0\text{ and }x = 1$.
\item Using integration, find the area of the region bounded by the line $2y = 5x + 7$, $x$-axis and the lins $x = 2\text{ and }x =8$.
\item Draw a rough sketch of the curve $y = \sqrt{x - 1}$ in the interval $[1, 5]$. Find the area under the curve and between the lines $x = 1\text{ and }x = 5$.
\item Determine the area under the curve $y = \sqrt{a^2 - x^2}$ included between the lines $x = 0\text{ and }x = a$
\item Find the area of the region bounded by $y = \sqrt{x}\text{ and }y = x$.
\item Find the area enclosed by the curve $y = - x^2$ and the straight lilne $x + y + 2 = 0$.
\item Find the area bounded by the curve $y = \sqrt{x}$, $x = 2y + 3$ in the first quadrant and $x$-axis.
\end{enumerate}
\textbf{Long Answer (L . A)}
\begin{enumerate}[resume]
\item Find the area of the region bounded by the curve $y^2 = 2x\text{ and }x^2 + y^2 = 4x$.
\item Find the area bounded by the curve $y = \sin x$ between $x = 0 \text{ and }x = {2\pi}$.
\item Find the area of region bounded by the triangle whose vertices are (-1, 1), (0, 5) and (3, 2), using integration.
\item Draw a rough sketch of the region ${(x, y) : y^2 \lessgtr 6ax\text{ and }x^2 + y^2 \lessgtr 16a^2}$.
\item Compute the area bouded by the line $x + 2y = 2$, $y - x = 1\text{ and }2x + y = 7$.
\item Find the area bonded by the lines $y = 4x + 5$, $y = 5 - x\text{ and }4y = x + 5$.
\item Find the area bounded by the curve $y = 2\cos x$ and the $x$-axis from $x = 0\text{ to }x = {2\pi}$.
\item Draw a  rough sketch of the given curve $y =1 + \abs{x + 1}$, $x = -3$, $x = 3$, $y = 0$, and find the area of the region bounded by them, using integration.
\end{enumerate}
\textbf{Objective Type Questions}

Choose the correct answer from the given four options in each of the Exercises 24 to 34.
\begin{enumerate}[resume]
\item The area of the region bounded by the $y - axis$, $ = \cos x\text{ and }y = \sin x$, $0 \leqslant x \leqslant \overline{2} $is
\begin{enumerate}
\item $\sqrt{2}$ sq units
\item $(\sqrt{2} + 1)$ sq units
\item $(\sqrt{2} - 1)$ sq units 
\item $(2\sqrt{2} - 1)$ sq units
\end{enumerate}
\item The area of the region bounded by the curve $x^2 = 4y$ and the straight line $x = 4y - 2$ is
\begin{enumerate}
\item $\frac{3}{8}$ sq units 
\item $\frac{5}{8}$ sq units
\item $\frac{7}{8}$ sq units 
\item $\frac{9}{8}$ sq units
\end{enumerate}
\item The area of the region bounded by the curve $y = \sqrt{16 - x^2}$ and $x$-axis is 
\begin{enumerate}
\item 8 sq units 
\item ${20\pi}$ sq units
\item ${16\pi}$ sq units
\item ${256\pi}$ sq units
\end{enumerate}
\item Area of the region in the first quadrant enclosed by the $x$-axis, the line $y = x$ and the circle $x^2 + y^2 = 32$ is 
\begin{enumerate}
\item ${16\pi}$ sq units 
\item ${4\pi}$ sq units
\item ${32\pi}$ sq units
\item ${24\pi}$ sq units
\end{enumerate}
\item Area of the region bounded bythe curve $y = \cos x$ between $x = 0\text{ and }{x =\pi}$ is 
\begin{enumerate}
\item 2 sq units
\item 4 sq units
\item 3 sq units
\item 1 sq units
\end{enumerate}
\item The area of the region bounded by parabola $y^2 = x$ and the straight line $2y = x$ is
\begin{enumerate}
\item $\frac{4}{3}$ sq units
\item 1 sq units
\item $\frac{2}{3}$ sq units 
\item $\frac{1}{3}$ sq units
\end{enumerate}
\item The area of the region bounded by the curve $y = \sin x$ between the ordinates $x = 0$, $x = \overline{2}$ and the $x$- axis is
\begin{enumerate}
\item 2 sq units
\item 4 sq units
\item 3 sq units
\item 1 sq units
\end{enumerate}
\item The area of the region bounded bythe ellipse $\frac{x^2}{25}+\frac{y^2}{16} = 1$ is
\begin{enumerate}
\item ${20\pi}$ sq units
\item ${20\pi^2}$ sq units 
\item ${16\pi^2}$ sq units
\item ${25\pi}$ sq units
\end{enumerate}
\item The area o thr region bounded by the circle $x^2 + y^2 = 1$ is
\begin{enumerate}
\item ${2\pi}$ sq units
\item ${\pi}$sq units
\item ${3\pi}$ sq units
\item ${4\pi}$ sq units
\end{enumerate}
\item The area of the region bounded by the curve $y = x + 1$ and the lines $x = 2\text{ and }x = 3$ is
\begin{enumerate}
\item $\frac{7}{2}$ sq units
\item $\frac{9}{2}$ sq units
\item $\frac{11}{2}$ sq units
\item $\frac{13}{2}$ sq units
\end{enumerate}   
\item The area of the region bounded by the curve $x = 2 + 3$ and th $y$ lines $y = 1\text{ and }y = - 1$ is
\begin{enumerate}
\item 4 sq units 
\item $\frac{3}{2}$ sq units
\item 6 sq units
\item 8 sq units
\end{enumerate}
\end{enumerate}
\end{document}
